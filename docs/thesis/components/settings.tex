% Included by MAIN.TEX
% Defines the settings for the CAMP report document

\renewcommand{\sectfont}{\normalfont \bfseries}        % Schriftart der Kopfzeile

% manipulate footer
\usepackage{scrpage2}
\pagestyle{scrheadings}
\ifoot[\footertext]{\footertext} % \footertext set in INFO.TEX
%\setkomafont{pagehead}{\normalfont\rmfamily}
\setkomafont{pagenumber}{\normalfont\rmfamily}

%% allow sophisticated control structures
\usepackage{ifthen}

% use Palatino as default font
\usepackage{palatino}

% enable special PostScript fonts
\usepackage{pifont}

% make thumbnails
% \usepackage{thumbpdf}

%to use the subfigures
\usepackage{subfigure}


\usepackage{colortbl}

\usepackage{multicol}

\usepackage{commath}

\usepackage{algorithm}
\usepackage{algpseudocode}
%black tirangle comments
% \algrenewcommand{\algorithmiccomment}[1]{\hfill$\blacktriangleright$ #1}
%normal arrow comments
\algrenewcommand{\algorithmiccomment}[1]{\hfill$\rightarrow$ #1}

%% show program code\ldots
%\usepackage{verbatim}
%\usepackage{program}

%% enable TUM symbols on title page
\usepackage{styles/tumlogo}


\usepackage{multirow}

%% use colors
\usepackage{color}

%%table package
\usepackage{tabu}

%% make fancy math
\usepackage{amsmath}
\usepackage{amsfonts}
\usepackage{amssymb}
\usepackage{textcomp}
\usepackage{yhmath} % fr die adots
%% mark text as preliminary

%% create an index
\usepackage{makeidx}

% for the program environment
\usepackage{float}
%for glossary
\usepackage[toc,  xindy]{glossaries}

%% load german babel package for german abstract
%\usepackage[german,american]{babel}
\usepackage[german,english]{babel}
\selectlanguage{english}

% use german characters as well
\usepackage[latin1]{inputenc}       % allow Latin1 characters

% use initals dropped caps - doesn't work with PDF
%\usepackage{dropping}
 %\usepackage[dvips]{dropping}

\usepackage{styles/shortoverview}
%----------------------------------------------------
%      Graphics and Hyperlinks
%----------------------------------------------------

%% check for pdfTeX
\ifx\pdftexversion\undefined
 %% use PostScript graphics
 \usepackage[dvips]{graphicx}

 \DeclareGraphicsExtensions{.eps,.epsi}
 \graphicspath{{figures/}{figures/review}}
 %% allow rotations
 \usepackage{rotating}
 %% mark pages as draft copies
 %\usepackage[english,all,light]{draftcopy}
 %% use hypertex version of hyperref
 \usepackage[hypertex,hyperindex=false,colorlinks=false]{hyperref}
\else %% reduce output size \pdfcompresslevel=9
 %% declare pdfinfo
 %\pdfinfo {
 %  /Title (my title)
 %  /Creator (pdfLaTeX)
 %  /Author (my name)
 %  /Subject (my subject	)
 %  /Keywords (my keywords)
 %}
 %% use pdf or jpg graphics
 \usepackage[pdftex]{graphicx}
 \DeclareGraphicsExtensions{.jpg,.JPG,.png,.pdf,.eps}
 \graphicspath{{figures/}}

 %% Load float package, for enabling floating extensions
 \usepackage{float}

 %% allow rotations
 \usepackage{rotating}
 %% use pdftex version of hyperref
 \usepackage[pdftex,colorlinks=true,linkcolor=red,citecolor=red,%
 anchorcolor=red,urlcolor=red,bookmarks=true,%
 bookmarksopen=true,bookmarksopenlevel=0,plainpages=false%
 bookmarksnumbered=true,hyperindex=false,pdfstartview=%
 ]{hyperref}
\fi

% default listing style

\usepackage{xcolor}
\usepackage{listings}

\definecolor{mygreen}{rgb}{0.073,0.573,0.073}
\definecolor{mygray}{rgb}{0.5,0.5,0.5}
\definecolor{mymauve}{rgb}{0.58,0,0.82}

\lstset{ %
	backgroundcolor=\color{white},   % choose the background color; you must add \usepackage{color} or \usepackage{xcolor}
	basicstyle=\footnotesize,        % the size of the fonts that are used for the code
	breakatwhitespace=false,         % sets if automatic breaks should only happen at whitespace
	breaklines=true,                 % sets automatic line breaking
	captionpos=b,                    % sets the caption-position to bottom
	commentstyle=\color{mygreen},    % comment style
	deletekeywords={...},            % if you want to delete keywords from the given language
	escapeinside={\%*}{*)},          % if you want to add LaTeX within your code
	extendedchars=true,              % lets you use non-ASCII characters; for 8-bits encodings only, does not work with UTF-8
	frame=tB,                    % adds a frame around the code (t_op single, B_ottom double)
	keepspaces=true,                 % keeps spaces in text, useful for keeping indentation of code (possibly needs columns=flexible)
	keywordstyle=\color{blue},       % keyword style
	language=C,  			               % the language of the code
	morekeywords={*,...},            % if you want to add more keywords to the set
	numbers=left,                    % where to put the line-numbers; possible values are (none, left, right)
	numbersep=5pt,                   % how far the line-numbers are from the code
	numberstyle=\tiny\color{mygray}, % the style that is used for the line-numbers
	rulecolor=\color{black},         % if not set, the frame-color may be changed on line-breaks within not-black text (e.g. comments (green here))
	showspaces=false,                % show spaces everywhere adding particular underscores; it overrides 'showstringspaces'
	showstringspaces=false,          % underline spaces within strings only
	showtabs=false,                  % show tabs within strings adding particular underscores
	stepnumber=1,                    % the step between two line-numbers. If it's 1, each line will be numbered
	stringstyle=\color{mymauve},     % string literal style
	tabsize=4,                       % sets default tabsize to 2 spaces
	%  title=\lstname                   % show the filename of files included with \lstinputlisting; also try caption instead of title
}





%% Fancy chapters
%\usepackage[Lenny]{fncychap}
%\usepackage[Glenn]{fncychap}
%\usepackage[Bjarne]{fncychap}

%\usepackage[avantgarde]{quotchap}

% set the bibliography style
%\bibliographystyle{styles/bauermaNum}
%\bibliographystyle{alpha}
\bibliographystyle{plain}
