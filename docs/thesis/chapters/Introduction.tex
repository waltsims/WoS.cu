
\chapter{Introduction}
\label{chapter:Introduction}


Harmonic \Gls{PDE} exhibit many interesting properties that
have long the been the subject of research.  Many of theses properties are well
understood \cite{Axler1992,Sheldon}.  Although there are methods to solve such
\Glspl{PDE} on a grid \cite{Bornemann}, discrete Monte Carlo Solvers have been shown
to be more efficient when only a
singular value in the domain is of interest, or in higher dimensions \cite{Bornemann, DeLaurentis, kakutani1944}.  They can also
be more reliable when severe gradients are present,
when finite-differences normally would fail \cite{DeLaurentis}. One example of such a case is
 turbulent fluid simulations, where the boundary behavior of a turbulent
fluid is still being researched.  These Monte Carlo Methods, which solve problems
through random inputs into a system, and subsequent evaluation of the result,
have long been a mathematical oddity, and have yet
to overtake more conventional solvers such as the finite differences and multi-grid method
methods.
\par
A major benefit of Monte Carlo methods, is their propensity for parallelism.  Often,
the the execution of individual random inputs into a system  and their subsequent evaluation
are independent, and can therefore be executed in parallel.  Through the exploitation of this
intrinsic parallelism Monte Carlo Methods might very well enjoy higher standings
in computational mathematics in the near future.
The current trend towards parallel computing has brought with it many new and exciting
technologies.  Along with multi-core processors, \Glspl{GPU} have moved on from
their humble beginnings as graphical co-processors.  Thanks to their execution
\Gls{simt} strategy and computational architecture, \Glspl{GPU} are being used for
high dimensional parallel tasks. Since so many have already
claimed that \Gls{RWoS} can and should be implemented in a highly parallel manner,
( \cite{DeLaurentis,Sabelfeld}, etc.), especially in high dimensions,  It is the goal
 of this paper to follow through on these claims.  Given the
current trend toward parallel computing, this work strives to implement
\Gls{RWoS}, a Monte-Carlo Solver for harmonic \Glspl{PDE} on one of the most modern
parallel architectures, and trends in parallel computing, the \Gls{GPU}.
\par
In this document we will show
one example of how the pairing of Monte Carlo methods with \Glspl{GPU} will enable
greater computational performance, and act as a proof of concept for future work.

%what this work will strive to do and what will be shown.


%practical analogy of how head dispersion from a point in a material or nuclear
%fusion also take a random path from a point and defuse randomly.
%physical analogy of the algorithm. integration of all paths could represent a single
% state at one time point.

% second analogy would be not being able to put sensors inside a system and want
%want to extrapolated data from boundary sensors
