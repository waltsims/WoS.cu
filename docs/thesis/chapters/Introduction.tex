\chapter{Introduction}
\label{chapter:Introduction}



Here starts the thesis with an introduction. Please use nice latex and bibtex entries\cite{Sabelfeld}. Do not spend time on formatting your thesis, but on its content.

\section{The Walking On Spheres algorithm}
%practical analogy of how head dispersion from a point in a material or nuclear
%fusion also take a random path from a point and defuse randomly.
%physical analogy of the algorithm. integration of all paths could represent a single
% state at one time point.

% second analogy would be not being able to put sensors inside a system and want
%want to extrapolated data from boundary sensors
\subsection{Harmonic Functions}
\subsubsection{Mean Value Property}
\subsubsection{The Maximum Principle}
\subsection{Brownian Motion}
\subsubsection{Excursion in Uniform Distributions on Spheres}
\subsection{Markov Chains}
\subsection{The Algorithm}



\section{The Basics of Parallel Computing}
In recent years, the convergence towards the physical limits of Moors law has lead to a
trend towards Parallel Computing\cite{Kumar}\cite{Markov}.  By most accounts, the
field of Parallel Computing conceived in the early 1958\cite{Gill}, but has
since bloomed into the State of the Art, for high-dimensional simulations in
High Performance Computing as hardware technology has caught up to the first
theoretical speculation of half a century prior. \par

Traditionally, the concept of Parallel Computing was simple; in order to improve
(reduce) computation time, one should separate independent steps of a given algorithm
and execute them concurrently upon unique processing units.  In doing so, for a given
problem, which traditionally would have taken time \textit{T}, in the optimal case,
the calculation could be completed in \textit{T/N} time, where \textit{N} describes
the number of unique processing units being utilized.  This observation was formalized
by Gene Amdahl in 1967\cite{Wilt}:
%
\begin{equation}
  \tag{Amdahl's Law}
  Speedup = S_{latency}(s)= \frac{1}{(r_{s} + \frac{r_{p}}{N})} = \frac{N}{N(1-r_{p})-r_{p}}
  \label{egn:Amdahl}
\end{equation}
%
where \textit{Speedup} embodies the factor of temporal acceleration a program
exhibits when additional computational resources are dedicated to it's execution.
 \textit{$r_{s}$} and \textit{$ r_{p} $} represent the ratio of the algorithm that must be executed serially or in parallel, respectively.  Amdahl's Law describes a problem of a fixed dimensin,
and the effects that more parallel computing resources have on execution time.
This strategy of optimization, in which the size of the problem is constant,
 is called Strong Scaling.
In extreme cases unlimited resources, the greatest speed up of a given algorithm
 is limited by it's serial component.
%
\begin{equation}
 \lim_{N\to\infty}  \frac{1}{(r_{s} + \frac{r_{p}}{N})} = \frac{1}{r_{s}} = \frac{1}{1-r_{p}}
\end{equation}
%
This description of optimal Speedup does not take into account issues of physical
distance of computing cores or the communication and facilitation overhead between
cores that can occur in real applications, which can lead to lower parallel efficiency.
Nevertheless, the model can serve as a useful tool when analyzing algorithmic
parallelization and and the greatest attainable Speedup for a given problem.\par

One further method of quantifying parallelism, is Gustafson's Law\cite{Gustafson}.
%
\begin{equation}
  \tag{Gustofson's Law}
  Speedup = S_{latency}(s) = r_{s} + (1 - r_{p})N
\end{equation}
%
This view of parallelism maintains that `` in practice, the problem
size scales with the number of processes'', and therefore
``assume[s] run time, not problem size, is constant''.
of the given problem. In other words, as the problem size grows, a proportional
amount of resources should be devoted to keep the execution time constant.
Both of these laws, assist in the quantification of parallel efficiency and
expected gains.

\subsection{GPU Hardware}

Modern computing architecture was first proposed by John Von Neumann in 1945.
This architecture is composed of three separate sections: the \Gls{CPU}, memory, and \Gls{I/O}.
Modern \Glspl{CPU} are comprised of an \Gls{ALU} comprised of electrical circuits
perform basic mathematical operations, such as addition, subtraction, multiplication and division,
registers, which store the data used for computation by the ALU and Cache, which is a more modern development to improve data locality.
\par
By design, \Glspl{CPU} are designed
to handle general purpose computing workloads.
An \Gls{ALU} computational
tasks on an individual datum. By the nature of this
generalized strategy, upcoming computation steps and which data are therefore required,
is not known to the CPU. If the ALU does not have the data necessary for a computation,
it has to wait for the data to be fetched from memory which can greatly increase computation time.
\par
Transfer times from \Gls{RAM}
are be orders of magnitude greater than the computation clock frequency of
a CPU.  This is due to the relatively large physical separation between the CPU and memory,
 the communication overhead needed to fetch the data, and
the nature of \Gls{DRAM}, which electrically refreshes data values at a designated rate.
These problems together are referred to the Von Neumann Bottleneck\cite{Backus}.
It is important therefore to work to predict which data will be required by upcoming operations.
The goal of modern CPU computation strategies, is to always ensure that the data
necessary for a computation is at hand, ready to be used, when requested by the \Gls{ALU}.
These strategies are called caching strategies.
\par
Cache is a bank of \Gls{SRAM}
that is etched on the same silicon wafer as the CPU itself.  This physical locality,
along with the higher read and write speeds compared to DRAM, lessen the transfer
time of data to the registers, and therefore increase the effective ability of the \Gls{ALU} to complete computations.
Due to the greater production cost of on-chip Cache, it's size is limited, and therefore
much research has been dedicated to optimizing cache load strategies of modern CPUs. %TODO cite
\par

It is important to grasp the strengths and weaknesses of modern \Glspl{CPU} to
recognize the computational benefits of \Glspl{GPU}
for modern computations problems.  As the name eludes, \Glspl{GPU} were originally developed
as dedicated computational units for 3D graphics\cite{Sanders}.  They are housed on the same
communication bus as the \Gls{CPU} and memory, and assigned image computational work,
to lessen the load of the CPU. %TODO: insert
The GPU specifically designed to handle workloads on large data sets (traditionally these
data sets were individual color values for digital images, and the workload as
the transformation of image data to modulate the output on a digital screen).
This heritage,
defined the strategy modern GPUs take towards computation.
The image data was was relatively high dimensional with intensity values for every pixel in the image.
Each pixel was at least three dimensional (one dimension per color value of red, green and or blue) and
some had a fourth value for alpha transparency.\cite{?}  These values would often all undergo
well defined modulation, that could be formulated in an individual instruction.
In the early 2000's these functions were named shaders, and later kernel functions.\cite{?}
Due to the independent nature of the execution of kernel functions, the GPUs have multiple, less robust
arithmetic units, that perform an individual instruction set on more well defined data.
Also, due to the dimensionality of the data being computed, \Glspl{GPU} have larger register\cite{?} sizes,
which allow for fewer context switches, and therefore greater \gls{ai}.

%TODO: expand
\subsection{GPU Architecture}
The GPU hardware architecture and terminology of said architecture varies from
vendor to vendor.  For the sake of uniformity, this document will adopt the
terminology employed by the manufacturer, NVIDIA.  To begin, in the following passages the CPU
will be referred to as the \textbf{host} while the GPU will be called \textbf{device}.
These are standard terminologies in GPU programing. GPUs are comprised of four major components:
\begin{description}
  \item[Host interface] moderates communications between the host and the device.
  Commands are dispatched to the necessary hardware, and synchronization between
  host and device is managed through this interface.
  \item[Copy Engine] manages the bidirectional transfer of data between the host
  and device.  Modern hardware can have 0-2 Copy Engines per device in order to
  manage the conversion of linear memory to CUDA arrays while saturating the PCI
  bus.  The independence of the Copy Engine(s) allows for concurrent data transfers
  and computation.
  \item[DRAM Interface] a memory interface with a bandwidth of up to 100Gb/sec.
  Some interfaces also support device side L2 caches usage.
  \item[TCPs \& GPCs] \underline{T}exture \underline{P}rocessing \underline{C}luster
  and \underline{G}raphics \underline{P}rocessing \underline{C}lusters are an assembly
  of \Glspl{SM} along with cache for computation.
\end{description}\cite{Wilt}

\subsubsection{Streaming Multiprocessors}
Each GPC contains a cluster of SM's.  The number of SM's per cluster varies by
model of GPU. Some examples can be found in %table bellow

%TODO list table with modern examples

Each SM is comprised of: \cite{Wilt}
\begin{itemize}
  \item execution units for 32-bit integer, as well as single and double precision arithmetic.
  \item Special function units for log/exp sin/cos and sqrt functions.
  \item a warp scheduler to coordinate the dispatch of threads.
  \item a constant cache to broadcast data to SM's.
  \item shared memory accessible by all threads.
  \item texture mapping hardware.
\end{itemize}

This basic overview of the architecture of the hardware is important to understand
how the the programming API maps to the available hardware.

%TODO: discuss streaming processing.

\section{Parallel Computing with CUDA}
\subsection{CUDA Software}

The CUDA parallel computing environment is a c++ language wrapper with built in
pragmas to interact with the NVIDIA hardware.  Although CUDA is modeled on the
hardware it runs on, it is not wholly analogous with the hardware architecture,
and higher level concepts have been added to aid in the implementation of CUDA code.
%TODO insert diagram of CUDA stack.
\par
The CUDA ecosystem stack is comprised of five layers, five of which are accessible, by the user.
These layers are the CUDA application itself, CUDA Libraries, CURA Runtime otherwise
known as CUDART, and the CUDA Driver.


\subsubsection{CUDA Libraries}

The highest level of this stack is occupied
of CUDA libraries. When looking to optimize code with CUDA, a
first step should be the implementation of CUDA Libraries.  These highly optimized
and user friendly libraries, such as CUBLAS and CUDA Thrust, allow the use of
CUDA hardware, with minimal understanding of the hardware or execution strategy.  These can prove
to be indispensable tools for beginner users or those looking to quickly and
efficiently prototype a problem. The author recommends their use as the first step
in any CUDA based undertaking. They provide the ability to create a quick proof
of concept for a project, as well as a highly optimized base line, should further
development be planned.
\subsubsection{CUDA Runtime}

Moving down the Stack, CUDA runtime provides an interface to the user which mirrors
standard C++ functionality such as malloc and memcpy.  These functions differentiate
themselves from host functions with the preface \textit{``cuda''} or \textit{``cu''}
e.g. cudaMalloc and cudaMemcpy.  Functions that are to be excited on the
device, are called kernels or kernel functions, which can be differentiated from
host functions via the triple angle bracket syntax, ``$<<< >>>$''.
%TODO example of basic CUDA kernel
Kernel parameters can be specified in these triple angle brackets.  These parameters
include number of blocks per grid, threads per blocks, size of shared memory and
stream, all topics witch shall be discussed later in this document. %TODO cite where.

%TODO describe blocks and threads
\subsubsection{CUDA Driver API}
The CUDA driver API is the interface to lower lever command sets used by the CUDA driver, which allows more detailed control
of kernel calls and memory allocation. Many of the same functionalities are possible,
but execution is more tedious and requires steps \cite{nvidia_cuda_driver}. This granularity of control can nevertheless be
helpful when tuning a program and striving for higher execution performance.
\subsubsection{CUDA Driver}

The CUDA driver is the lowest level of integration between the host and the device
and facilitates communication between the \Gls{GPU} and the rest of the computer via the bus. %TODO: define Bus.


\subsection{Compilation and Execution}

%TODO: Make diagram. CUDA book page 58

CUDA programs are written in CUDA flavor C++.  The CUDA wrapper of C++ is the
compiler driver nvcc, which can compile, link and execute CUDA code.  This process occurs
 in three main steps. First,
files containing CUDA pragmas are assigned the .cu ending format, and split upon
compilation into device and host sections.  Second, these files are subsequently
compiled separately into to host and device machine code respectively.
Lastly, both compilates are merged into a single executable, to be run on the host.
\par
Device machine code, is referred to as \hyphenation{mi-cro-code} microcode, which is compiled for a specific
NVIDIA device model's architecture. In order to be able to execute the same executable
on multiple \Glspl{GPU} and to allow backwards compatibility of CUDA features, CUDA code is
first compiled into \Gls{PTX}, a pseudo-assembly language.  \Gls{PTX} is then
further compiled in to device microcode and writen into a CUDA binary file called .cubin. This can either happen offline, at compile time,
or online, at runtime, in a \gls{jit} fashion at compiled time.  By default, both
the microcode for the local device at compilation and the generalized \gls{PTX} compilates are added to the to the executable file.
this way, should the executable be run on a different device, the \gls{PTX} can
be recompiled appropriately with \Gls{jit} compilation, and executed on the a desired device\cite{Wilt}.

\subsection{Programming CUDA}
%separate address space from CPU --> mem copies
%shared vs. global memory
\subsubsection{Memories}
GPU memory does not benefit from features such as virtual
memory or paging, as one might expect from CPU memory. instead devices have an
address space completely separate from the host address space, where every address is assigned a physical space in memory.
In many cases memory consists of a dedicated chip on the device that is controlled
by the memory controller.  This memory must be managed by the user when programming in CUDA.
\par
Not all memories are created equal when working with GPU's.  Similarly to \Glspl{CPU},
different memories have varying strengths and weaknesses.  By preallocating memory for a variable,
that is either read only, or read and written often, one can greatly increase the
performance of ones program.   %TODO: CUDA attempts to avoid lazy allocation--> performance gains.
There are four main types of CUDA memories.
\begin{description}
  \item [Global memory] Memory on the Device with bandwidth of up to 100 Gb/sec.
  Global memory is allocated dynamically by the user via cudaMalloc() and cudaFree() functions.
  Dynamic allocation is normally performed in host code, due to large performance short fall
  when implemented in a kernel call.
  %TODO: coalescing?
  \item [Texture memory] Constant memory with dedicated texture access pattern controller.

  \item [Constant memory] Read only memory on device with higher read speeds.
  \item [Shared memory] User managed cache located on every \Gls{SM}.
  \item [Host memory] global memory used for program stack used for overflow variables
  \item [registers] Memory units on \Gls{SM} used to store thread local variables,
                    and variables for current computation.
\end{description}

%blocks and threads
\subsubsection{Blocks and Threads}

%kernel syntax
\subsubsection{Kernels}



%streams
\subsubsection{Streams}
